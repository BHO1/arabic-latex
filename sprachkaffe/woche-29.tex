%% -*- mode: LaTeX; TeX-PDF-mode: nil; LaTeX-command: "xelatex " -*-
\documentclass[a4paper]{article}

%\usepackage{multicol}
\usepackage{fontspec}
\usepackage{polyglossia}
\usepackage{marginfit}
\usepackage{abraces}
%\setmainlanguage{english}
%\setotherlanguage{arabic}

\setmainlanguage{arabic}
\setotherlanguage{english}

%\newfontfamily\englishfont{Palatino}
% \setmainlanguage{arabic}
\newfontfamily\arabicfont[Script=Arabic,Scale=2]{Amiri}
%
%\usepackage{refenums}
%\usepackage{gb4e}	   
%\usepackage{covington}

\begin{document}


﷽


%\newfontfamily\arabicfont[Script=Arabic,Scale=2]{KacstBook}
\begin{center}
  اللُغت العربية \textenglish{sprachkaffe}  في مدينة الربط
\end{center}

\section{\textenglish{19.7.2018}}

\begin{enumerate}
\item  قَرَأَ الوَلَدُ القِصَةَ.
  \begin{enumerate}
  \item \textenglish{plural masculin:} قَرَأَ اَلْأَوْلادُ القِصَّةَ.
  \item \textenglish{dual masculin} قَرَأَ الوَلَدَانِ القِصَّة.
  \end{enumerate}
\item الطَالِبُ يَتَعَلَّمُ العَرَبِيَّةَ مُنْذُ سَنَةٍ
  \begin{enumerate}
  \item \textenglish{plural masculin}: الطُّلابُ يَتَعَلَّمُونَ العَرَبِيَّةَ مُنْذُ سَنَة.
  \item \textenglish{verbal sentence}: يَتْعَلَمُ الطّالِبُ العَرَبِيَّةَ مُنْذُ سَنَة.
  \end{enumerate}
\item أَحَبَّ أَخِي السَّفَرَ وَ دِراسَةَ اللُّغاتِ.
\begin{enumerate}
  \item \textenglish{plural masculin}: أَحَبَّ إِخوتي  السَّفَرَ وَ دِراسَةَ اللُّغاتِ. 
  \item \textenglish{plural feminin}: أحَبَّتْ أخْتِي  السَّفَرَ وَ دِراسَةَ اللُّغاتِ. 
  \end{enumerate}

  \textenglish{ad verbal Sentence: when the subject is not exlicit in phrase, the verb take that function and is accorded in plural/singular}
  
\item لا تَسْتَطِيعينَ  أَنْ تَحْفَظِي كلّ الكلمات الجديدة. %\margin{\textenglish{blabla}}
\begin{enumerate}
  \item \textenglish{plural feminin}: لا تسْتَطِيعينِ أَنْ تَحْفَطْنَ كْلّ  الكلمات الجديدة. 
  \item \textenglish{plural masculin}: لا يَسْتَطِيعونَ  أَنْ تَحْفَطْنَ كْلّ  الكلمات الجديدة.
  \end{enumerate}
\item صَدِيقَتِي مَا أَرادَتْ أَنْ تَذْهَبَ مَعِي إِلَى السِينِمَا
\begin{enumerate}
  \item \textenglish{plural feminin}: صَدِيقاتي ما أَرإدْنَ أَنْ يَذْهَبْنَ.
  \item \textenglish{plural masculin}: أَصْديقاءِي ما أَرادُوا أَن يَذْهَبُوا.
  \end{enumerate}
\end{enumerate}  
\subsection{comments}

(  إِخوة + ي ) = إِخوتي





\section{\textenglish{19.7.2018}}
\subsection{\textenglish{singular, plural}}
استعمل المصادر لإكمال الجمل:


\begin{enumerate}
\item  مَرْيَمُ تُحِبُ الذَّهَابَ إِلَى مصر مَعَ عَائِلَتِهَا (تذهب)
\item  أَنْدْرِيَا تُرِيدُ حِفْظَ كُلَ الكَلِمَاتِ الجَدِيدَةِ. (تحفظ)
  \item  أَنَا أُحِبُ قِرَاءَةَ الكُتُبِ الجَيدَةِ. (يقرأ) 
\item  التَّدْرِيسُ فِي sprachcaffe مُمْتَازَة. (يدرس)
\item أُحِبُ السَكَنَ فِي بِرْلِينْ. (يسكن)
  \item  أُرِيدُ أَكْلَ الطَاجِينَ المَغْرِبِي دَائِماً. (يأكل)
\end{enumerate}

\subsection{\textenglish{Almasdar}}
\begin{enumerate}
\item أَحِبُّ أَنْ أَسْكُنَ فِي مُرَّاكُش.
  \begin{itemize}
  \item أُحِبُّ السَّكَنَ فِي مُرّاكُش.
  \end{itemize}
\item هِيَ تُرِيدُ أَنْ تَدْرُسَ العَرَبِيَّة.
  \begin{itemize}
  \item هَيَ تُرِيدُ دِرَاسةَ العَرَبِيَّة.
  \end{itemize}

\item لاَ أَسْتَطِيعُ أَنْ أَذْهَبَ مَعَكُمْ
  \begin{itemize}
  \item لاَ أَسْتَطِيعُ الذَّهَابَ مَعَكُمْ
  \end{itemize}
\item هُوَ يَتَعَلم العَرَبِيّة لِيَعْمَلَ فِي مِصْر.
  \begin{itemize}
  \item هُوَ يَتَعَلم العَرَبِيّة لِلْعَمَلِ فِي مِصْر.
  \end{itemize}
\item أَخِي لاَ يُرِيدُ كِتَابَة الوَاجِبْ
  \begin{itemize}
  \item أَخِي لاَ يُرِيدُ أَنْ يَكْتُبَ الوَاجِب
  \end{itemize}
\item ذَهَبْنَا إِلَى المَطْعَم لِنَأْكُلَ الطَاجِين المَغْرِبِيّْ.
  \begin{itemize}
  \item ذَهَبْنَا إِلَى المَطْعَم لِأَكْل الطَاجِين.
  \end{itemize}
\end{enumerate}

\textenglish{rule}إضافة



\section{18.7.2018}
\subsection{اكمل هذه الجمل بـ "ليس"}
"ليس"   Fill in the blacks with the appropriate conjugation of
\begin{enumerate}
\item \dots عِنْدَ وَالِدِي زُمَلاَءَ يَعْمَلُونَ فِي هَذِهِ المَدِينَة الكَبِيرَة.
\item هَذِهِ المُدُن \dots  جَمِلَةِ مِثْلَ مَدِينَة مُرَّاكُش.
\item الُّلاَبُ فِي هَدِهِ المَدْرَسَةِ \dots   لَطِفِينَ مَعِي، لِذَلِكَ أَنَا لاَ أُحِبُ أَنْ أَدْرُسَ مَعَهُم فِي نَفْسِ المَدْرَسَة.
\item زَوْجَتِي \dots طَالِبَة فِي هَذْهِ الجَامِعَة.
\item هَذَا الرَّجُ  \dots  وَالِدِي.  
\end{enumerate}


\subsection{Change the sentences from plural to singular or singular to plural :}

\begin{enumerate}
\item الطَّالِبَات اليَابَانِيَّات يَدْرُسْنَ كَثِيراً
\item أَمْسَ الوَلَدَ لَعِبَ كُرَةَ القَدَم مَعَ صَدِيقِهِ.
\item يَوْمَ الإِثَنَيْنَ  أَنَا سَوْفَ احْفَظُ كَلِمَةِ جَدِيدَة.
\item فِي مَكْتَبَة الحَيِ كُتُبُ قَدِيمَة وَ نَادِرَة.
\item فِي مَدْرَسَتِي طُلاَّبَ وَ أَسَاتِذَة مِنْ جِنْسِيَّات مُخْتَلِفَة.
\item إِخْوَاتُهَا لاَ يُحِبُّون المُوسِيقَى الكَلاَسِيكِيَّة.
\item أُحِبُّ أَنْ أُسَافِرَ إِلَى بَلَد عَرَبِّي وَ أَزُورَ المَدِينَة الجَميِلَة.
\item النِّسَاء المَغْرِبِيَّات جَمِيلاَت.
\item الأُسْرَة المَغْرِبِيَّة مُخْتَلِفَة عَنْ الأُسْرَة الإِيطَالِيّة.
\end{enumerate}

\section{17.7.2018}
\subsection{Correct the mistakes}

\begin{enumerate}
\item فَاطِمَة لاَ مَرِيض.
\item أُخْتُنَا عُمْرُهُ 26 سَنَوَات.
\item البَيْت صَغِيرة.
\item أَسْكُنُ إِلَى شُقَّة جَميل.
\item وَالِدَتِي تَعْمَل مُمَرِّض فِي مُسْتَشْفَى.
\item يُوسُف تَجْلَسِين فِي نَفْس الكُرْسِي.
\item هَذَا صُوَرَة جَمِيلَة.
\item اليَوْم الجَوْ لاَ مُشْمِس.
\item شَاهَدْتُ بُيُوت فِي المَدِينَة القَدِيم.
\item الطَّالِبَات الفَرَنْسِّيَة مَا يَدْرُسْنَ اليَوْم.
\item أَخِي تَعْمَل فِي الشَّرِكَة مَعَ
\item غَداً أَتَكَلَّمُ مَعَ الأُسَاتِذَة وَ الطُّلاَب بِالعَرَبِيَّة.
\item لِمَاذَا صَدِيقَات سَعِيدَة اليَوْم.
\end{enumerate}

\subsection{أحول من المفرد إلى الجمع:}
\begin{enumerate}
\item هَذَا الكِتَاب قَدِيم جِداً
\item تَدْرُسَ الطَّالِبَة الفَرَنْسِيّة فِي هَدِه الجَامِعَة الكَبِيرَة.
\item الأُسْتَاذَة فِي مَدْرَسَتَنَا طَيِّبَة.
\item يُتَرْجِمُ المُتَرْجِم مِن وَ إِلَى اللَّغَة العَرَبِيَة والفرنسية.
\item الشَارِع فِي مَدِينَة الرِبَاط مُزْدَحِم.
\item الطَّاوِلَة فِي هَذَا الَّف صَغِيرَة.
\item فِي طُفُولَتِي أَلْعَبُ مَعَ صَديقَتِي.
\end{enumerate}

\begin{enumerate}
\item أكتبوا الافعال بالشكل المناسب
\item هُمَا مَا................فِي هَذَا الشَّارِع. (سكن)
\item الطِّفْلاَن ............ كُرَة القَدَم. (لعب)
\item كُلَ ضَبَاح هُوَ وَ زَوْجَتُه ...............الفُطُور مَعاً. (أكل)
\item الطَّالِبَتَان ........اللَّغَةَ العَرَبِيَّة الأُسْبُوع المَاضِي. (درس)
\item عَادَةً أَنْتُمَا ..........طَعَامَ العَشَاء. (طبخ)
\item أَحْمَد وَ سَارَة مَا ...............فِي نَفْس الشَّرِكَة. (عمل)
\item مَرْيم وَ سَارة ............... كُلَّ الكَلِمَات الجَدِيدَة (حفظ)
\item هَلْ أَنْتُمَا .......... الطَّالِب الجَدِيد فِي صَفِنَا. (عرف)

\end{enumerate}
% \begin{example}
%    \gll\textarabic{ قَرَأَ} \textarabic{ الوَلَدُ} \textarabic{ القِصَةَ.}
%  read.3Sg child.definit.Sg.nominativ story.definit.Sg.acc 
%  \glt The child reads a story 
%  \glend
% \end{example}



% grammaire:
% \begin{exe}
% \ex 
% \gll Кот ест сметану\\
% cat.NOM eat.3.SG.PRS sour-cream.ACC\\
% \trans `The cat eats sour cream'
% \end{exe}


\end{document}