%% -*- mode: LaTeX; TeX-PDF-mode: nil; LaTeX-command: "xelatex " -*-
\documentclass[a4paper]{article}
%\usepackage{media9}
\usepackage{hyperref}
%\usepackage{attachfile}
\usepackage{multicol}
\usepackage{polyglossia}
%\usepackage[hmargin=1cm, vmargin=1cm]{geometry}
\setmainlanguage{arabic}
%\newfontfamily\arabicfont[Path=/home/olivier/arab/latex/amiri-0.109/, Script=Arabic]{amiri}
%\newfontfamily\arabicfont[Script=Arabic,WordSpace=2]{Amiri}
%\newfontfamily\arabicfont[Script=Arabic]{Amiri Bold}
\newfontfamily\arabicfont[Script=Arabic]{Amiri}
%\newfontfamily\arabicfont[Script=Arabic]{Lateef}
% \newfontfamily\arabicfont[Script=Arabic]{Scheherazade Bold}
%Scheherazade-Regular
% \newfontfamily\arabicfont[Path=/home/olivier/arab/latex/amiri-0.109/, UprightFont = *-regular, BoldFont = *-bold,Script=Arabic]{amiri}
%\newfontfamily\arabicfont[Script=Arabic]{amiri}

%\usepackage{arabxetex}

%\pagestyle{empty}

\XeTeXinterchartokenstate=1
\newXeTeXintercharclass\confb % connect back
\newXeTeXintercharclass\conb  % connect front back
\newXeTeXintercharclass\alif  % alif
\newXeTeXintercharclass\lam   % lam

\XeTeXcharclass `\ي=\confb 
\XeTeXcharclass `\ئ=\confb
\XeTeXcharclass `\ه=\confb
\XeTeXcharclass `\ش=\confb
\XeTeXcharclass `\س=\confb
\XeTeXcharclass `\ق=\confb
\XeTeXcharclass `\ف=\confb
\XeTeXcharclass `\غ=\confb
\XeTeXcharclass `\ع=\confb
\XeTeXcharclass `\ض=\confb
\XeTeXcharclass `\ص=\confb
\XeTeXcharclass `\ن=\confb
\XeTeXcharclass `\م=\confb
\XeTeXcharclass `\ك=\confb
\XeTeXcharclass `\ظ=\confb
\XeTeXcharclass `\ط=\confb
\XeTeXcharclass `\خ=\confb
\XeTeXcharclass `\ح=\confb
\XeTeXcharclass `\ج=\confb
\XeTeXcharclass `\ث=\confb
\XeTeXcharclass `\ت=\confb
\XeTeXcharclass `\ب=\confb

\XeTeXcharclass `\ل=\lam

\XeTeXcharclass `\ا=\alif
\XeTeXcharclass `\أ=\alif
\XeTeXcharclass `\إ=\alif
\XeTeXcharclass `\آ=\alif

\XeTeXcharclass `\و=\conb
\XeTeXcharclass `\ؤ=\conb
\XeTeXcharclass `\ذ=\conb
\XeTeXcharclass `\د=\conb
\XeTeXcharclass `\ز=\conb
\XeTeXcharclass `\ر=\conb
\XeTeXcharclass `\ة=\conb

 \XeTeXinterchartoks \confb \confb = {\kashida{}}
 \XeTeXinterchartoks \lam \lam     = {\kashida{}}
 \XeTeXinterchartoks \confb \alif  = {\kashida{}}
 \XeTeXinterchartoks \confb \lam   = {\kashida{}}
 \XeTeXinterchartoks \lam \confb   = {\kashida{}}
 \XeTeXinterchartoks \lam \conb    = {\kashida{}}
 \XeTeXinterchartoks \confb \conb  = {\kashida{}}

 \newlength\kashidaheight
 \setlength\kashidaheight{\heightof{\large{\textarabic{ـ}}}}
 % \setlength\kashidaheight{\heightof{\char"0640}}
 % \setlength\kashidaheight{\heightof{\textarabic{۔}}}
 \newlength\kashidadepth
 \setlength\kashidadepth{\depthof{\large{\textarabic{ـ}}}}
 % \setlength\kashidadepth{\depthof{\textarabic{\char"0640}}}
 % \setlength\kashidadepth{\depthof{\textarabic{۔}}}

 
\newcommand\kashida[1]{\char"200D
	    \nobreak\leaders
		\hrule height \kashidaheight depth \kashidadepth
		\hskip 0pt plus 100 pt
	    \nobreak\char"200D}

	   % \renewcommand\kashida{\char"200D} % turn off kashida

	   \frenchspacing
\title{كليلة ودمن}
\author{Olivier Büchel}
\date{\today}
	   

	\begin{document}
\maketitle
﷽

demonstartiv pronoun

\begin{enumerate}
\item  هَذا الرَجُل يَعمَلُ في شَرِكة.
\item  السَّيِّدة الدسْنابِيّة تَتَكْلَّمُ 5 لُغات.
\item هَذِهِ الجارَة كَريمَة جِداً.
\item  داّءِماً بَعْد الدّرْس, الطَّالِب يدْجعُ  إلى بيتِهِ.

\end{enumerate}

1. هَذا الرَجُل يَعمَلُ في شَرِكة.
2. السَّيِّدة الدسْنابِيّة تَتَكْلَّمُ 5 لُغات.
3. هَذِهِ الجارَة كَريمَة جِداً.
4. داّءِماً بَعْد الدّرْس, الطَّالِب يرْجعُ  إلى بيتِهِ.


\section{الناسك واللصوص }
%\fontsize{10pt}{21pt}\selectfont
\href{run:/home/olivier/arab/kalilawadimna/audio/B12-MuntherA.Younes-.mp3}{اِسْمَعْ}
\href{run:/home/olivier/arab/kalilawadimna/Tales\_from\_Kalila\_Wa\_Dimna\_al3arabiya.pdf\#page=118}{اِقْرَأْ}
%\href{run:/home/olivier/arab/kalilawadimna/Tales\_from\_Kalila\_Wa\_Dimna\_al3arabiya.pdf::118}{اِقْرَأْ}
%\href{file:///home/olivier/arab/kalilawadimna/Tales\_from\_Kalila\_Wa\_Dimna\_al3arabiya.pdf\#page=118}{اِقْرَأْ}



\begin{large}
{\begin{multicols}{3}\noindent
  اشترى ناسك ماعِزاً ومشى به الى بيته. وفي الطريق راه لُصوص واتّفقوا على سرقة الماعز. فذهبوا الناسك وقال واحد منهم : »~ما هذا الكلب الَّذي معك؟~« ثم قال الثاني: »~ هذا ليس ناسكاً, فالناسك لا يَقود كلباً.« واستمرّوا في ذلك يسألون الناسك عن الحيوان الذي معه ويقولون انّه كلب, حتى صدّق الناسك كلامهم وقال في نفسه إنّ الرجل الذي باعه الماعز باعه كلباً وليس ماعزاً, وقد سَحَرَه حتى ظَنّ أنّ الكلب ماعز. ثمّ ترك الماعز فأخذه اللصوص.
\end{multicols}  }
\end{large}
\end{document}