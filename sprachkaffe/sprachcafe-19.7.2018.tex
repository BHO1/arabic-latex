%% -*- mode: LaTeX; TeX-PDF-mode: nil; LaTeX-command: "xelatex " -*-
\documentclass[a4paper]{article}

%\usepackage{multicol}
\usepackage{fontspec}
\usepackage{polyglossia}

%\setmainlanguage{english}
%\setotherlanguage{arabic}

\setmainlanguage{arabic}
\setotherlanguage{english}

%\newfontfamily\greekfont{Gentium}
% \setmainlanguage{arabic}
\newfontfamily\arabicfont[Script=Arabic]{Amiri}

%\usepackage{gb4e}	   
\usepackage{covington}

\begin{document}


% قَرَأَ الوَلَدُ القِصَةَ.
% الطَالِبُ يَتَعَلَّمُ العَرَبِيَّةَ مُنْذُ سَنَةٍ
% أَحَبَّ أَخِي السَّفَرَ وَ دِراسَةَ اللُّغاتِ
% لا تَسْتَطِيعينَ  أَنْ تَحْفَظِي كلّ الكلمات الجديدة
% صَدِيقَتِي مَا أَرادَتْ أَنْ تَذْهَبَ مَعِي إِلَى السِينِمَا

% استعمل المصادر لإكمال الجمل:
% 1- مَرْيَمُ تُحِبُ الذَّهَابَ إِلَى مصر مَعَ عَائِلَتِهَا (تذهب)
% 2- أَنْدْرِيَا تُرِيدُ حِفْظَ كُلَ الكَلِمَاتِ الجَدِيدَةِ. (تحفظ)
% 3- أَنَا أُحِبُ قِرَاءَةَ الكُتُبِ الجَيدَةِ. (يقرأ)
% 4- التَّدْرِيسُ فِي sprachcaffe مُمْتَازَة. (يدرس)
% 5- أُحِبُ السَكَنَ فِي بِرْلِينْ. (يسكن)
% 6- أُرِيدُ أَكْلَ الطَاجِينَ المَغْرِبِي دَائِماً. (يأكل)

% أَحِبُّ أَنْ أَسْكُنَ فِي مُرَّاكُش.
% أُحِبُّ السَّكَنَ فِي مُرّاكُش.
% هِيَ تُرِيدُ أَنْ تَدْرُسَ العَرَبِيَّة.
% هَيَ تُرِيدُ دِرَاسةَ العَرَبِيَّة.
% لاَ أَسْتَطِيعُ أَنْ أَذْهَبَ مَعَكُمْ
% لاَ أَسْتَطِيعُ الذَّهَابَ مَعَكُمْ
% هُوَ يَتَعَلم العَرَبِيّة لِيَعْمَلَ فِي مِصْر.
% هُوَ يَتَعَلم العَرَبِيّة لِلْعَمَلِ فِي مِصْر.
% أَخِي لاَ يُرِيدُ كِتَابَة الوَاجِبْ
% أَخِي لاَ يُرِيدُ أَنْ يَكْتُبَ الوَاجِب
% ذَهَبْنَا إِلَى المَطْعَم لِنَأْكُلَ الطَاجِين المَغْرِبِيّْ.
% ذَهَبْنَا إِلَى المَطْعَم لِأَكْل الطَاجِين.




18/07/2018
اكمل هذه الجمل بـ "ليس"
"ليس"   Fill in the blacks with the appropriate conjugation of
………… عِنْدَ وَالِدِي زُمَلاَءَ يَعْمَلُونَ فِي هَذِهِ المَدِينَة الكَبِيرَة.
هَذِهِ المُدُن .....................جَمِلَةِ مِثْلَ مَدِينَة مُرَّاكُش.
الُّلاَبُ فِي هَدِهِ المَدْرَسَةِ ..............لَطِفِينَ مَعِي، لِذَلِكَ أَنَا لاَ أُحِبُ أَنْ أَدْرُسَ مَعَهُم فِي نَفْسِ المَدْرَسَة.
زَوْجَتِي ..............طَالِبَة فِي هَذْهِ الجَامِعَة.
هَذَا الرَّجُل .....وَالِدِي.

18/07/2018
Change the sentences from plural to singular or singular to plural :
الطَّالِبَات اليَابَانِيَّات يَدْرُسْنَ كَثِيراً
أَمْسَ الوَلَدَ لَعِبَ كُرَةَ القَدَم مَعَ صَدِيقِهِ.
يَوْمَ الإِثَنَيْنَ  أَنَا سَوْفَ احْفَظُ كَلِمَةِ جَدِيدَة.
فِي مَكْتَبَة الحَيِ كُتُبُ قَدِيمَة وَ نَادِرَة.
فِي مَدْرَسَتِي طُلاَّبَ وَ أَسَاتِذَة مِنْ جِنْسِيَّات مُخْتَلِفَة.
إِخْوَاتُهَا لاَ يُحِبُّون المُوسِيقَى الكَلاَسِيكِيَّة.
أُحِبُّ أَنْ أُسَافِرَ إِلَى بَلَد عَرَبِّي وَ أَزُورَ المَدِينَة الجَميِلَة.
النِّسَاء المَغْرِبِيَّات جَمِيلاَت.
الأُسْرَة المَغْرِبِيَّة مُخْتَلِفَة عَنْ الأُسْرَة الإِيطَالِيّة.

17/07/2018
Correct the mistakes
فَاطِمَة لاَ مَرِيض.
أُخْتُنَا عُمْرُهُ 26 سَنَوَات.
البَيْت صَغِيرة.
أَسْكُنُ إِلَى شُقَّة جَميل.
وَالِدَتِي تَعْمَل مُمَرِّض فِي مُسْتَشْفَى.
يُوسُف تَجْلَسِين فِي نَفْس الكُرْسِي.
هَذَا صُوَرَة جَمِيلَة.
اليَوْم الجَوْ لاَ مُشْمِس.
شَاهَدْتُ بُيُوت فِي المَدِينَة القَدِيم.
الطَّالِبَات الفَرَنْسِّيَة مَا يَدْرُسْنَ اليَوْم.
أَخِي تَعْمَل فِي الشَّرِكَة مَعَ
غَداً أَتَكَلَّمُ مَعَ الأُسَاتِذَة وَ الطُّلاَب بِالعَرَبِيَّة.
لِمَاذَا صَدِيقَات سَعِيدَة اليَوْم.

16/7/2018
أحول من المفرد إلى الجمع:
هَذَا الكِتَاب قَدِيم جِداً
تَدْرُسَ الطَّالِبَة الفَرَنْسِيّة فِي هَدِه الجَامِعَة الكَبِيرَة.
الأُسْتَاذَة فِي مَدْرَسَتَنَا طَيِّبَة.
يُتَرْجِمُ المُتَرْجِم مِن وَ إِلَى اللَّغَة العَرَبِيَة والفرنسية.
الشَارِع فِي مَدِينَة الرِبَاط مُزْدَحِم.
الطَّاوِلَة فِي هَذَا الَّف صَغِيرَة.
فِي طُفُولَتِي أَلْعَبُ مَعَ صَديقَتِي.

16/07/2018
أكتبوا الافعال بالشكل المناسب
هُمَا مَا................فِي هَذَا الشَّارِع. (سكن)
الطِّفْلاَن ............ كُرَة القَدَم. (لعب)
كُلَ ضَبَاح هُوَ وَ زَوْجَتُه ...............الفُطُور مَعاً. (أكل)
الطَّالِبَتَان ........اللَّغَةَ العَرَبِيَّة الأُسْبُوع المَاضِي. (درس)
عَادَةً أَنْتُمَا ..........طَعَامَ العَشَاء. (طبخ)
أَحْمَد وَ سَارَة مَا ...............فِي نَفْس الشَّرِكَة. (عمل)
مَرْيم وَ سَارة ............... كُلَّ الكَلِمَات الجَدِيدَة (حفظ)
هَلْ أَنْتُمَا .......... الطَّالِب الجَدِيد فِي صَفِنَا. (عرف)

% \begin{example}
%    \gll\textarabic{ قَرَأَ} \textarabic{ الوَلَدُ} \textarabic{ القِصَةَ.}
%  read.3Sg child.definit.Sg.nominativ story.definit.Sg.acc 
%  \glt The child reads a story 
%  \glend
% \end{example}



% grammaire:
% \begin{exe}
% \ex 
% \gll Кот ест сметану\\
% cat.NOM eat.3.SG.PRS sour-cream.ACC\\
% \trans `The cat eats sour cream'
% \end{exe}


\end{document}